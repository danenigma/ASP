\documentclass[a4paper,11pt]{article}
\usepackage{color}
\usepackage{graphicx}
\usepackage{subcaption}
\usepackage{amsmath}
\usepackage{tikz}
\usepackage{listings}
\definecolor{codegreen}{rgb}{0,0.6,0}
\definecolor{codegray}{rgb}{0.5,0.5,0.5}
\definecolor{codepurple}{rgb}{0.58,0,0.82}
\definecolor{backcolour}{rgb}{0.95,0.95,0.92}
 
\lstdefinestyle{mystyle}{
    backgroundcolor=\color{backcolour},   
    commentstyle=\color{codegreen},
    keywordstyle=\color{magenta},
    numberstyle=\tiny\color{codegray},
    stringstyle=\color{codepurple},
    basicstyle=\footnotesize,
    breakatwhitespace=false,         
    breaklines=true,                 
    captionpos=b,                    
    keepspaces=true,                 
    numbers=left,                    
    numbersep=5pt,                  
    showspaces=false,                
    showstringspaces=false,
    showtabs=false,                  
    tabsize=2
}
 
\lstset{style=mystyle}
\usetikzlibrary{automata,positioning}

\graphicspath{ {images/} }
\begin{document}
\title{\color{red}CARNEGIE MELLON UNIVERSITY\\
APPLIED STOCHASTIC PROCESSES  (COURSE 18-751)\\
HOMEWORK 13}
\author{Daniel Marew}
\date{\today}
\clearpage\maketitle

\thispagestyle{empty}
\newpage
I collaborated with :\\
\hspace*{6cm}
Nebyou Yismaw\\
\hspace*{6cm}
Daniel    Nkemelu\\
\hspace*{6cm}
Agatha Niwomugizi
\thispagestyle{empty}
\newpage
\clearpage
\setcounter{page}{1}
\section*{2.b}
\begin{figure}[h]
  \hspace*{-5cm}
   \includegraphics[scale=0.45]{q2_b_1}
   \caption{$h_n$}\label{fig:q3}
\end{figure}
\section*{2.c}
\begin{figure}[h]
  \hspace*{-5cm}
   \includegraphics[scale=0.45]{q2_c_1}
   \caption{random sequence of impulses}\label{fig:q3}
\end{figure}
\begin{figure}[h]
  \hspace*{-5cm}
   \includegraphics[scale=0.45]{q2_c_2}
   \caption{random sequence of impulses}\label{fig:q3}
\end{figure}
\begin{figure}[h]
  \hspace*{-5cm}
   \includegraphics[scale=0.45]{q2_c_3}
   \caption{random sequence of impulses}\label{fig:q3}
\end{figure}
Comment on the apparent bandwidth of the system in all three cases\\
The bandwidth increases as we consider more and more parts of the impulse response(filter) i.e as c increases.
\clearpage
\newpage
\section*{2.d}
\begin{figure}[h]
  \hspace*{-5cm}
   \includegraphics[scale=0.45]{q_2_1}
   \caption{random sequence of impulses}\label{fig:q3}
\end{figure}
\begin{figure}[h]
  \hspace*{-5cm}
   \includegraphics[scale=0.45]{q_2_2}
   \caption{random sequence of pulses}\label{fig:q3}
\end{figure}
\begin{figure}[h]
  \hspace*{-5cm}
   \includegraphics[scale=0.45]{q_2_3}
   \caption{$SXX(\phi)$ $c=2$}\label{fig:q3}
\end{figure}

\begin{figure}[h]
  \hspace*{-5cm}
   \includegraphics[scale=0.45]{q_2_4}
   \caption{$SXX(\phi)$ $c=5$}\label{fig:q3}
\end{figure}

\begin{figure}[h]
  \hspace*{-5cm}
   \includegraphics[scale=0.45]{q_2_5}
   \caption{$SXX(\phi)$ $c=10$}\label{fig:q3}
\end{figure}
\clearpage
\newpage
\begin{appendix}
\section*{Code Appendix}
\lstinputlisting[language=Octave]{hm13.m} 
\end{appendix}
\end{document}